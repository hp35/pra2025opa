%
% File: backopo/theory/bwopatheory.tex [plain TeX code]
% Created:     March 30, 2025
% Last change: April 6, 2025
%
% Derivation of the equations governing backward OPA in chiral media,
% including description of pulsed propagation and dispersion.
%
% Copyright (C) 2025, Fredrik Jonsson
%
\input macros/epsf.tex
\input macros/eplain.tex
\font\ninerm=cmr9
\font\twentyrm=cmr12 at 20 truept
\font\twelvesc=cmcsc10 at 12 truept
\input amssym % to get the {\Bbb E} font (strikethrough E)
\def\captionwide{\advance\leftskip by 60pt
  \advance\rightskip by 60pt}
\def\document #1 {\hsize=150mm\hoffset=4.6mm\vsize=230mm\voffset=7mm
  \topskip=0pt\baselineskip=12pt\parskip=0pt\leftskip=0pt\parindent=15pt
  \headline={\ifnum\pageno>1\ifodd\pageno\rightheadline\else\leftheadline\fi
    \else\hfill\fi}
  \def\rightheadline{\tenrm{\it #1}
    \hfil{\it\date}}
  \def\leftheadline{\tenrm{\it\date}
    \hfil{\it #1}}
  \noindent~\vskip-60pt\hskip-40pt{\epsfbox{macros/UU_logo_color.eps}}
  \vskip-42pt\hfill\vbox{\hbox{{\it\author}}
  \hbox{{\it\date}}}\vskip 36pt
  \centerline{\twelvesc #1}
  \vskip 24pt\noindent}
\def\section #1 {\bigskip\goodbreak\noindent{\bf #1}
  \par\nobreak\smallskip\noindent}
\def\subsection #1 {\bigskip\goodbreak\noindent{\it #1}
  \par\nobreak\smallskip\noindent}
\def\iint{\mathop{\int\kern-8pt\int}}
\def\iiint{\mathop{\int\kern-8pt\int\kern-8pt\int}}
\def\oiint{\mathop{\int\kern-8pt\int\kern-13.2pt{\bigcirc}}}
\def\Re{\mathop{\rm Re}\nolimits} % real part
\def\Im{\mathop{\rm Im}\nolimits} % imaginary part
\def\Tr{\mathop{\rm Tr}\nolimits} % quantum mechanical trace
\def\fourier{\mathop{\frak F}\nolimits}
\def\eqq{\mathop{\vbox{\hbox{\hskip2pt?}\vskip-6pt\hbox{=}}}}

%
% Blackboard bold fonts.
%
\font\mbb=msbm10 
\newfam\bbb
\textfont\bbb=\mbb

%
% Define in which way we want displayed equation numbers to appear in
% the text. In this case, it is preferred to have the equation numbers
% as one single running index, rather than the form otherwise commonly
% found format used in books, (<chapter number>.<equation number>).
%
\def\eqconstruct#1{#1}

%
% Define in which way we want displayed subequation numbers
% to appear in the text.
%
\newcount\subref
\def\eqsubreftext#1#2{%
  \subref = #2           % The space stops a <number>.
  \advance\subref by 96  % `a' is character code 97.
  #1{\rm\char\subref}%
}

%
% Define the 'boxit' macro from D.E. Knuths "The TeXbook, Exercise 21.3.
%
\def\boxit#1{\vbox{\hrule\hbox{\vrule\kern3pt
  \vbox{\kern3pt#1\kern3pt}\kern3pt\vrule}\hrule}}

\def\date{April 6, 2025}
\def\author{Fredrik Jonsson}
\document{Theory of backward optical parametric amplification in chiral media}
\vskip24pt
\noindent
In the following analysis, we will keep the model for pulsed wave propagation,
as this easily is expressed for continuous waves by simply dropping the first
and second order time derivatimes. The notations follow the separate document
{\it Pulse propagation in chiral optical parametric processes}, as well as the
manuscript {\it Pulsed optical parametric amplification in chiral media} by
Fredrik Jonsson, Christos Flytzanis and Govind Agrawal, as submitted to
J.~Opt.~Soc.~Am. B in March 2025.
\bigskip
\centerline{\epsfxsize=90mm\epsfbox{metapost/fig-01.eps}}
\medskip
\noindent
{\bf Figure~1}.
Schematic of the setup for backward-wave optical parametric amplification, in
which the pump and signal waves are launched at $z=0$, and where the idler is
created in the opposite direction. The medium is in a quasi phase matching
configuration with a periodic modulation of the sign of the nonlinear coupling
coefficient, with spatial period of $\Lambda$.

\section{Conventions for the fields}
The natural time-harmonic oscillation of the quasi-monochromatic electric field
is for the complex-valued amplitude of the total electric field taken as
$$
  {\bf E}({\bf r},t)=\sum_{\omega_{\sigma}}{\rm Re}[
           {\bf E}_{\omega_{\sigma}}({\bf r},t)\exp(-i\omega_{\sigma}t)].
$$
We express this field in the circularly polarized base vectors ${\bf e}_+$
(left circular polarization, LCP) and ${\bf e}_-$ (right circular polarization,
RCP),
$$
  {\bf e}_+={{1}\over{\sqrt{2}}}({\bf e}_x+i{\bf e}_y),\qquad
  {\bf e}_-={{1}\over{\sqrt{2}}}({\bf e}_x-i{\bf e}_y).
  \eqdef{eq:circbasis10}
$$
in terms of which the LCP and RCP components of the total electric field are
projected by using the orthogonality conditions
${\bf e}^*_{\pm}\cdot{\bf e}_{\pm}=1$ and ${\bf e}^*_{\pm}\cdot{\bf e}_{\mp}=0$,
to yield\numberedfootnote{We may equally well inverse these relations and
  express the fields in the Cartesian coordinate system in terms of these
  circularly polarized components, as
  $$
    E^x_{\omega}={{1}\over{\sqrt{2}}}(E^+_{\omega}+E^-_{\omega}),\qquad
    E^y_{\omega}={{i}\over{\sqrt{2}}}(E^+_{\omega}-E^-_{\omega}).
  $$}
$$
  E^{\pm}_{\omega}={\bf e}^*_{\pm}
    \cdot(E^+_{\omega}{\bf e}_++E^-_{\omega}{\bf e}_-).
$$
Finally, the natural spatially harmonic oscillation of the field of forward
and backward traveling components are formulated by
$$
  E^{\pm}_{\omega_k}(z,t)=
    E^{f\pm}_{\omega_k}(z,t)\exp(i(\omega n_k/c)z)
    +E^{b\mp}_{\omega_k}(z,t)\exp(-i(\omega n_k/c)z),
  \eqdef{eq:waveeq110}
$$
for $k=1,2,3$ for the idler, signal and pump, respectively, in which we should
note that any corrections to the wave vectors due to, say, circular dichroism
or birefringence will show up in the wave equations to follow.

\section{Wave equations for backward optical parametric amplification}
From the separate document {\it Pulse propagation in chiral optical parametric
processes}, using the notation therein, the relevant equations of propagation
are, for the forward and backward propagating components of the idler at
angular frequency~$\omega_1$,
$$
  \eqalign{
    &\bigg\{
       {{\partial E^{f\pm}_{\omega_1}}\over{\partial z}}
       \pm i{{\omega^2_1 \gamma_1}\over{2c^2}} E^{f\pm}_{\omega_1}
       +\Big(
         (k'_1\mp a'_1) {{\partial}\over{\partial t}}
         +{{i}\over{2}}(k''_1\mp b''_1) {{\partial^2}\over{\partial t^2}}
       \Big) E^{f\pm}_{\omega_1}
    \bigg\}
    \exp\Big(i{{\omega_1 n_1}\over{c}}z\Big)\cr
    &\quad+\bigg\{
       -{{\partial E^{b\mp}_{\omega_1}}\over{\partial z}}
       \mp i{{\omega^2_1 \gamma_1}\over{2c^2}} E^{b\mp}_{\omega_1}
       +\Big(
         (k'_1\pm a'_1) {{\partial}\over{\partial t}}
         +{{i}\over{2}}(k''_1\pm b''_1) {{\partial^2}\over{\partial t^2}}
       \Big) E^{b\mp}_{\omega_1}
    \bigg\}
    \exp\Big(-i{{\omega_1 n_1}\over{c}}z\Big)
    \cr&\hskip80pt
      = i{{\omega_1}\over{2cn_1}}
      \Big(p_1 \pm iq_1 {{\partial}\over{\partial z}}\Big)
      \Big[
        E^{f\mp}_{\omega_3} E^{f\pm*}_{\omega_2}
          \exp\Big(
            i\Big({{\omega_3 n_3}\over{c}}-{{\omega_2 n_2}\over{c}}\Big)z
          \Big)
    \cr&\hskip200pt
       +E^{f\mp}_{\omega_3} E^{b\mp*}_{\omega_2}
          \exp\Big(
            i\Big({{\omega_3 n_3}\over{c}}+{{\omega_2 n_2}\over{c}}\Big)z
          \Big)
    \cr&\hskip200pt
       +E^{b\pm}_{\omega_3} E^{f\pm*}_{\omega_2}
          \exp\Big(
            -i\Big({{\omega_3 n_3}\over{c}}+{{\omega_2 n_2}\over{c}}\Big)z
          \Big)
    \cr&\hskip200pt
       +E^{b\pm}_{\omega_3} E^{b\mp*}_{\omega_2}
          \exp\Big(
            -i\Big({{\omega_3 n_3}\over{c}}-{{\omega_2 n_2}\over{c}}\Big)z
          \Big)
      \Big],\cr
  }
  \eqdef{eq:waveeq160}
$$
while the wave equation for the signal at angular frequency~$\omega_2$ is
$$
  \eqalign{
    &\bigg\{
       {{\partial E^{f\pm}_{\omega_2}}\over{\partial z}}
       \pm i{{\omega^2_2 \gamma_2}\over{2c^2}} E^{f\pm}_{\omega_2}
       +\Big(
         (k'_2\mp a'_2) {{\partial}\over{\partial t}}
         +{{i}\over{2}}(k''_2\mp b''_2) {{\partial^2}\over{\partial t^2}}
       \Big) E^{f\pm}_{\omega_2}
    \bigg\}
    \exp\Big(i{{\omega_2 n_2}\over{c}}z\Big)\cr
    &\quad+\bigg\{
       -{{\partial E^{b\mp}_{\omega_2}}\over{\partial z}}
       \mp i{{\omega^2_2 \gamma_2}\over{2c^2}} E^{b\mp}_{\omega_2}
       +\Big(
         (k'_2\pm a'_2) {{\partial}\over{\partial t}}
         +{{i}\over{2}}(k''_2\pm b''_2) {{\partial^2}\over{\partial t^2}}
       \Big) E^{b\mp}_{\omega_2}
    \bigg\}
    \exp\Big(-i{{\omega_2 n_2}\over{c}}z\Big)
    \cr&\hskip80pt
      = i{{\omega_2}\over{2cn_2}}
      \Big(p_2 \pm iq_2 {{\partial}\over{\partial z}}\Big)
      \Big[
        E^{f\mp}_{\omega_3} E^{f\pm*}_{\omega_1}
          \exp\Big(
            i\Big({{\omega_3 n_3}\over{c}}-{{\omega_1 n_1}\over{c}}\Big)z
          \Big)
    \cr&\hskip200pt
       +E^{f\mp}_{\omega_3} E^{b\mp*}_{\omega_1}
          \exp\Big(
            i\Big({{\omega_3 n_3}\over{c}}+{{\omega_1 n_1}\over{c}}\Big)z
          \Big)
    \cr&\hskip200pt
       +E^{b\pm}_{\omega_3} E^{f\pm*}_{\omega_1}
          \exp\Big(
            -i\Big({{\omega_3 n_3}\over{c}}+{{\omega_1 n_1}\over{c}}\Big)z
          \Big)
    \cr&\hskip200pt
       +E^{b\pm}_{\omega_3} E^{b\mp*}_{\omega_1}
          \exp\Big(
            -i\Big({{\omega_3 n_3}\over{c}}-{{\omega_1 n_1}\over{c}}\Big)z
          \Big)
      \Big],\cr
  }
  \eqdef{eq:waveeq162}
$$
and, finally, the wave equation for the pump at angular frequency~$\omega_3$ is
$$
  \eqalign{
    &\bigg\{
       {{\partial E^{f\mp}_{\omega_3}}\over{\partial z}}
       \mp i{{\omega^2_3 \gamma_3}\over{2c^2}} E^{f\mp}_{\omega_3}
       +\Big(
         (k'_3\pm a'_3) {{\partial}\over{\partial t}}
         +{{i}\over{2}}(k''_3\pm b''_3) {{\partial^2}\over{\partial t^2}}
       \Big) E^{f\mp}_{\omega_3}
    \bigg\}
    \exp\Big(i{{\omega_3 n_3}\over{c}}z\Big)\cr
    &\quad+\bigg\{
       -{{\partial E^{b\pm}_{\omega_3}}\over{\partial z}}
       \pm i{{\omega^2_3 \gamma_3}\over{2c^2}} E^{b\pm}_{\omega_3}
       +\Big(
         (k'_3\mp a'_3) {{\partial}\over{\partial t}}
         +{{i}\over{2}}(k''_3\mp b''_3) {{\partial^2}\over{\partial t^2}}
       \Big) E^{b\pm}_{\omega_3}
    \bigg\}
    \exp\Big(-i{{\omega_3 n_3}\over{c}}z\Big)
    \cr&\hskip80pt
      = i{{\omega_3}\over{2cn_3}}
      \Big(p_3 \mp iq_3 {{\partial}\over{\partial z}}\Big)
      \Big[
        E^{f\pm}_{\omega_1} E^{f\pm}_{\omega_2}
          \exp\Big(
            i\Big({{\omega_1 n_1}\over{c}}+{{\omega_2 n_2}\over{c}}\Big)z
          \Big)
    \cr&\hskip200pt
       +E^{f\pm}_{\omega_1} E^{b\mp}_{\omega_2}
          \exp\Big(
            i\Big({{\omega_1 n_1}\over{c}}-{{\omega_2 n_2}\over{c}}\Big)z
          \Big)
    \cr&\hskip200pt
       +E^{b\mp}_{\omega_1} E^{f\pm}_{\omega_2}
          \exp\Big(
            -i\Big({{\omega_1 n_1}\over{c}}-{{\omega_2 n_2}\over{c}}\Big)z
          \Big)
    \cr&\hskip200pt
       +E^{b\mp}_{\omega_1} E^{b\mp}_{\omega_2}
          \exp\Big(
            -i\Big({{\omega_1 n_1}\over{c}}+{{\omega_2 n_2}\over{c}}\Big)z
          \Big)
      \Big].\cr
  }
  \eqdef{eq:waveeq164}
$$
\vfill\eject

\section{Projecting out linear gyrotropy from the envelopes}
The wave equations~\eqref{eq:waveeq160}--\eqref{eq:waveeq164} may be simplified
considerably by only considering close to phase-matched components, say by
multiplying by $\exp(i\omega_j n_j z/c)$ and averaging over a few spatial
periods. However, before projecting out these matched terms, we may in
Eqs.~\eqref{eq:waveeq160}--\eqref{eq:waveeq164} straight away, from the
appearance of the terms with coefficients $\gamma_1$, $\gamma_2$ and $\gamma_3$,
find that the gyrotropic nature of the forward and backward traveling wave
envelopes can be separated by the {\it ansatz}
$$
  \eqalignno{
    E^{f\pm}_{\omega_k}(z,t)&=A^{f\pm}_{\omega_k}(z,t)
      \exp\Big(\mp i{{\omega^2_k\gamma_k}\over{2c^2}}z\Big),
        \eqdefn{eq:waveeq170}&\eqsubdef{eq:waveeq170a}\cr
    E^{b\mp}_{\omega_k}(z,t)&=A^{b\mp}_{\omega_k}(z,t)
      \exp\Big(\mp i{{\omega^2_k\gamma_k}\over{2c^2}}z\Big),
        &\eqsubdef{eq:waveeq170b}\cr
  }
$$
with $k=1,2,3$ for the idler, signal and pump, respectively. This way, carrying
out the second step in the reduction of the coupled set of nonlinear PDE:s, the
terms containing the gyration coefficients $\gamma_k$ will be cancelled.
Notice that in separating the field into forward and backward traveling
components, the backward traveling LCP/RCP components will be connected
to the {\it conjugated} basis vectors ${\bf e}^*_{\pm}$, hence the altered
order of ``$\mp$'' in ``$E^{b\mp}_{\omega_k}$''.
Thus, by inserting the ansatz of Eq.~\eqref{eq:waveeq170} into
Eq.~\eqref{eq:waveeq160}, we for the idler at angular frequency~$\omega_1$
obtain
$$
  \eqalign{
    &\bigg\{
       {{\partial A^{f\pm}_{\omega_1}}\over{\partial z}}
       +(k'_1\mp a'_1) {{\partial A^{f\pm}_{\omega_1}}\over{\partial t}}
       +{{i}\over{2}}(k''_1\mp b''_1)
            {{\partial^2 A^{f\pm}_{\omega_1}}\over{\partial t^2}}
    \bigg\}
    \exp\Big(i\Big({{\omega_1 n_1}\over{c}}
             \mp {{\omega^2_1\gamma_1}\over{2c^2}}\Big)z\Big)
    \cr&\qquad
    +\bigg\{
       -{{\partial A^{b\mp}_{\omega_1}}\over{\partial z}}
       +(k'_1\pm a'_1) {{\partial A^{b\mp}_{\omega_1}}\over{\partial t}}
       +{{i}\over{2}}(k''_1\pm b''_1)
            {{\partial^2 A^{b\mp}_{\omega_1}}\over{\partial t^2}}
    \bigg\}
    \exp\Big(-i\Big({{\omega_1 n_1}\over{c}}z
             \pm {{\omega^2_1\gamma_1}\over{2c^2}}\Big)z\Big)
    \cr&\hskip40pt
      = i{{\omega_1}\over{2cn_1}}
      \Big(p_1 \pm iq_1 {{\partial}\over{\partial z}}\Big)
      \Big[
        A^{f\mp}_{\omega_3} A^{f\pm*}_{\omega_2}
          \exp\Big(
            i\Big({{\omega_3 n_3}\over{c}}-{{\omega_2 n_2}\over{c}}\Big)z
            \pm i\Big({{\omega^2_3\gamma_3}\over{2c^2}}
                       +{{\omega^2_2\gamma_2}\over{2c^2}}\Big)z
          \Big)
    \cr&\hskip140pt
       +A^{f\mp}_{\omega_3} A^{b\mp*}_{\omega_2}
          \exp\Big(
            i\Big({{\omega_3 n_3}\over{c}}+{{\omega_2 n_2}\over{c}}\Big)z
            \pm i\Big({{\omega^2_3\gamma_3}\over{2c^2}}
                       +{{\omega^2_2\gamma_2}\over{2c^2}}\Big)z
          \Big)
    \cr&\hskip140pt
       +A^{b\pm}_{\omega_3} A^{f\pm*}_{\omega_2}
          \exp\Big(
            -i\Big({{\omega_3 n_3}\over{c}}+{{\omega_2 n_2}\over{c}}\Big)z
            \pm i\Big({{\omega^2_3\gamma_3}\over{2c^2}}
                       +{{\omega^2_2\gamma_2}\over{2c^2}}\Big)z
          \Big)
    \cr&\hskip140pt
       +A^{b\pm}_{\omega_3} A^{b\mp*}_{\omega_2}
          \exp\Big(
            -i\Big({{\omega_3 n_3}\over{c}}-{{\omega_2 n_2}\over{c}}\Big)z
            \pm i\Big({{\omega^2_3\gamma_3}\over{2c^2}}
                       +{{\omega^2_2\gamma_2}\over{2c^2}}\Big)z
          \Big)
      \Big],\cr
  }
  \eqdef{eq:waveeq180}
$$
while we for the signal at angular frequency~$\omega_2$ obtain
$$
  \eqalign{
    &\bigg\{
       {{\partial A^{f\pm}_{\omega_2}}\over{\partial z}}
       +(k'_2\mp a'_2) {{\partial A^{f\pm}_{\omega_2}}\over{\partial t}}
       +{{i}\over{2}}(k''_2\mp b''_2)
            {{\partial^2 A^{f\pm}_{\omega_2}}\over{\partial t^2}}
    \bigg\}
    \exp\Big(i\Big({{\omega_2 n_2}\over{c}}
             \mp {{\omega^2_2\gamma_2}\over{2c^2}}\Big)z\Big)
    \cr&\qquad
    +\bigg\{
       -{{\partial A^{b\mp}_{\omega_2}}\over{\partial z}}
       +(k'_2\pm a'_2) {{\partial A^{b\mp}_{\omega_2}}\over{\partial t}}
       +{{i}\over{2}}(k''_2\pm b''_2)
            {{\partial^2 A^{b\mp}_{\omega_2}}\over{\partial t^2}}
    \bigg\}
    \exp\Big(-i\Big({{\omega_2 n_2}\over{c}}z
             \pm {{\omega^2_2\gamma_2}\over{2c^2}}\Big)z\Big)
    \cr&\hskip40pt
      = i{{\omega_2}\over{2cn_2}}
      \Big(p_2 \pm iq_2 {{\partial}\over{\partial z}}\Big)
      \Big[
        A^{f\mp}_{\omega_3} A^{f\pm*}_{\omega_1}
          \exp\Big(
            i\Big({{\omega_3 n_3}\over{c}}-{{\omega_1 n_1}\over{c}}\Big)z
            \pm i\Big({{\omega^2_3\gamma_3}\over{2c^2}}
                       +{{\omega^2_1\gamma_1}\over{2c^2}}\Big)z
          \Big)
    \cr&\hskip140pt
       +A^{f\mp}_{\omega_3} A^{b\mp*}_{\omega_1}
          \exp\Big(
            i\Big({{\omega_3 n_3}\over{c}}+{{\omega_1 n_1}\over{c}}\Big)z
            \pm i\Big({{\omega^2_3\gamma_3}\over{2c^2}}
                       +{{\omega^2_1\gamma_1}\over{2c^2}}\Big)z
          \Big)
    \cr&\hskip140pt
       +A^{b\pm}_{\omega_3} A^{f\pm*}_{\omega_1}
          \exp\Big(
            -i\Big({{\omega_3 n_3}\over{c}}+{{\omega_1 n_1}\over{c}}\Big)z
            \pm i\Big({{\omega^2_3\gamma_3}\over{2c^2}}
                       +{{\omega^2_1\gamma_1}\over{2c^2}}\Big)z
          \Big)
    \cr&\hskip140pt
       +A^{b\pm}_{\omega_3} A^{b\mp*}_{\omega_1}
          \exp\Big(
            -i\Big({{\omega_3 n_3}\over{c}}-{{\omega_1 n_1}\over{c}}\Big)z
            \pm i\Big({{\omega^2_3\gamma_3}\over{2c^2}}
                       +{{\omega^2_1\gamma_1}\over{2c^2}}\Big)z
          \Big)
      \Big],\cr
  }
  \eqdef{eq:waveeq182}
$$
\vfill\eject\noindent
and finally for the pump at angular frequency~$\omega_3$,
$$
  \eqalign{
    &\bigg\{
       {{\partial A^{f\mp}_{\omega_3}}\over{\partial z}}
       +(k'_3\pm a'_3) {{\partial A^{f\mp}_{\omega_3}}\over{\partial t}}
       +{{i}\over{2}}(k''_3\pm b''_3)
            {{\partial^2 A^{f\mp}_{\omega_3}}\over{\partial t^2}}
    \bigg\}
    \exp\Big(i\Big({{\omega_3 n_3}\over{c}}
             \pm {{\omega^2_3\gamma_3}\over{2c^2}}\Big)z\Big)
    \cr&\qquad
    +\bigg\{
       -{{\partial A^{b\pm}_{\omega_3}}\over{\partial z}}
       +(k'_3\mp a'_3) {{\partial A^{b\pm}_{\omega_3}}\over{\partial t}}
       +{{i}\over{2}}(k''_3\mp b''_3)
            {{\partial^2 A^{b\pm}_{\omega_3}}\over{\partial t^2}}
    \bigg\}
    \exp\Big(-i\Big({{\omega_3 n_3}\over{c}}z
             \mp {{\omega^2_3\gamma_3}\over{2c^2}}\Big)z\Big)
    \cr&\hskip40pt
      = i{{\omega_3}\over{2cn_3}}
      \Big(p_3 \mp iq_3 {{\partial}\over{\partial z}}\Big)
      \Big[
        A^{f\pm}_{\omega_1} A^{f\pm}_{\omega_2}
          \exp\Big(
            i\Big({{\omega_1 n_1}\over{c}}+{{\omega_2 n_2}\over{c}}\Big)z
            \mp i\Big({{\omega^2_1\gamma_1}\over{2c^2}}
                       +{{\omega^2_2\gamma_2}\over{2c^2}}\Big)z
          \Big)
    \cr&\hskip140pt
       +A^{f\pm}_{\omega_1} A^{b\mp}_{\omega_2}
          \exp\Big(
            i\Big({{\omega_1 n_1}\over{c}}-{{\omega_2 n_2}\over{c}}\Big)z
            \mp i\Big({{\omega^2_1\gamma_1}\over{2c^2}}
                       +{{\omega^2_2\gamma_2}\over{2c^2}}\Big)z
          \Big)
    \cr&\hskip140pt
       +A^{b\mp}_{\omega_1} A^{f\pm}_{\omega_2}
          \exp\Big(
            -i\Big({{\omega_1 n_1}\over{c}}-{{\omega_2 n_2}\over{c}}\Big)z
            \mp i\Big({{\omega^2_1\gamma_1}\over{2c^2}}
                       +{{\omega^2_2\gamma_2}\over{2c^2}}\Big)z
          \Big)
    \cr&\hskip140pt
       +A^{b\mp}_{\omega_1} A^{b\mp}_{\omega_2}
          \exp\Big(
            -i\Big({{\omega_1 n_1}\over{c}}+{{\omega_2 n_2}\over{c}}\Big)z
            \mp i\Big({{\omega^2_1\gamma_1}\over{2c^2}}
                       +{{\omega^2_2\gamma_2}\over{2c^2}}\Big)z
          \Big)
      \Big].\cr
  }
  \eqdef{eq:waveeq184}
$$

\section{Separating out closely phase-matched terms}
We will now narrow down the algebra to yield a configuration in which we have
forward traveling pump and signal waves, while having a backward traveling
idler wave. In this case, the idler wave at angular frequency~$\omega_1$ is
from Eq.~\eqref{eq:waveeq180} obtained as
$$
  \eqalign{
   &-{{\partial A^{b\mp}_{\omega_1}}\over{\partial z}}
       +(k'_1\pm a'_1) {{\partial A^{b\mp}_{\omega_1}}\over{\partial t}}
       +{{i}\over{2}}(k''_1\pm b''_1)
            {{\partial^2 A^{b\mp}_{\omega_1}}\over{\partial t^2}}\cr
    &\hskip90pt
    =i{{\omega_1 p_1}\over{2cn_1}} A^{f\mp}_{\omega_3} A^{f\pm*}_{\omega_2}
        \exp(i(k_3-k_2+k_1)z\pm i(\beta_3+\beta_2+\beta_1)z)\cr
    &\hskip90pt
    +i{{\omega_1 p_1}\over{2cn_1}} A^{f\mp}_{\omega_3} A^{b\mp*}_{\omega_2}
        \exp(i(k_3+k_2+k_1)z\pm i(\beta_3+\beta_2+\beta_1)z)\cr
    &\hskip90pt
    +i{{\omega_1 p_1}\over{2cn_1}} A^{b\pm}_{\omega_3} A^{f\pm*}_{\omega_2}
        \exp(-i(k_3+k_2-k_1)z\pm i(\beta_3+\beta_2+\beta_1)z)\cr
    &\hskip90pt
    +i{{\omega_1 p_1}\over{2cn_1}} A^{b\pm}_{\omega_3} A^{b\mp*}_{\omega_2}
        \exp(-i(k_3-k_2-k_1)z\pm i(\beta_3+\beta_2+\beta_1)z),\cr
  }
  \eqdef{eq:waveeq210}
$$
where we for the sake of simplicity ignored the nonlocal correction to the
nonlinear coupling coefficients. In similar, the forward traveling signal
wave at $\omega_2$ is obtained as
$$
  \eqalign{
    &{{\partial A^{f\pm}_{\omega_2}}\over{\partial z}}
       +(k'_2\mp a'_2) {{\partial A^{f\pm}_{\omega_2}}\over{\partial t}}
       +{{i}\over{2}}(k''_2\mp b''_2)
            {{\partial^2 A^{f\pm}_{\omega_2}}\over{\partial t^2}}\cr
    &\hskip90pt
    =i{{\omega_2 p_2}\over{2cn_2}} A^{f\mp}_{\omega_3} A^{f\pm*}_{\omega_1}
        \exp(i(k_3-k_1-k_2)z\pm i(\beta_3+\beta_2+\beta_1)z)\cr
    &\hskip90pt
    +i{{\omega_2 p_2}\over{2cn_2}} A^{f\mp}_{\omega_3} A^{b\mp*}_{\omega_1}
        \exp(i(k_3+k_1-k_2)z\pm i(\beta_3+\beta_2+\beta_1)z)\cr
    &\hskip90pt
    +i{{\omega_2 p_2}\over{2cn_2}} A^{b\pm}_{\omega_3} A^{f\pm*}_{\omega_1}
        \exp(-i(k_3+k_1+k_2)z\pm i(\beta_3+\beta_2+\beta_1)z)\cr
    &\hskip90pt
    +i{{\omega_2 p_2}\over{2cn_2}} A^{b\pm}_{\omega_3} A^{b\mp*}_{\omega_1}
        \exp(-i(k_3-k_1+k_2)z\pm i(\beta_3+\beta_2+\beta_1)z),\cr
  }
  \eqdef{eq:waveeq220}
$$
and, finally, the forward traveling pump wave at $\omega_3$ as
$$
  \eqalign{
    &{{\partial A^{f\mp}_{\omega_3}}\over{\partial z}}
       +(k'_3\pm a'_3) {{\partial A^{f\mp}_{\omega_3}}\over{\partial t}}
       +{{i}\over{2}}(k''_3\pm b''_3)
            {{\partial^2 A^{f\mp}_{\omega_3}}\over{\partial t^2}}\cr
    &\hskip90pt
    =i{{\omega_3 p_3}\over{2cn_3}} A^{f\pm}_{\omega_1} A^{f\pm}_{\omega_2}
        \exp(i(k_1+k_2-k_3)z\mp i(\beta_3+\beta_2+\beta_1)z)\cr
    &\hskip90pt
    +i{{\omega_3 p_3}\over{2cn_3}} A^{f\pm}_{\omega_1} A^{b\mp}_{\omega_2}
        \exp(i(k_1-k_2-k_3)z\mp i(\beta_3+\beta_2+\beta_1)z)\cr
    &\hskip90pt
    +i{{\omega_3 p_3}\over{2cn_3}} A^{b\mp}_{\omega_1} A^{f\pm}_{\omega_2}
        \exp(-i(k_1-k_2+k_3)z\mp i(\beta_3+\beta_2+\beta_1)z)\cr
    &\hskip90pt
    +i{{\omega_3 p_3}\over{2cn_3}} A^{b\mp}_{\omega_1} A^{b\mp}_{\omega_2}
        \exp(-i(k_1+k_2+k_3)z\mp i(\beta_3+\beta_2+\beta_1)z).\cr
  }
  \eqdef{eq:waveeq230}
$$
In these expressions, we defined
$$
  k_j\equiv {{\omega_j n_j}\over{c}},\qquad
  \beta_j\equiv {{\omega^2_j\gamma_j}\over{2c^2}},
  \eqdef{eq:waveeq240}
$$
with, as previously, $j=1,2,3$ denoting the idler, signal and pump,
respectively.
Notice the way the phase matching in Eqs.~\eqref{eq:waveeq230} appear,
with the phase mismatch from the electric-dipolar parts occurring with
various combinations of the wave vector magnitudes $k_j$, while the gyrotropic,
non-local contributions all appear as the sum $\beta_1+\beta_2+\beta_3$.

If we now focus on terms in which the phase matching yields
$$
  k_3-k_2+k_1 \approx 0,
  \eqdef{eq:waveeq250}
$$
we will in the right-hand sides of Eqs.~\eqref{eq:waveeq210}--%
\eqref{eq:waveeq230} only keep one of the terms, in this case reducing the
set of equations to
$$
  \eqalignno{
   &-{{\partial A^{b\mp}_{\omega_1}}\over{\partial z}}
       +(k'_1\pm a'_1) {{\partial A^{b\mp}_{\omega_1}}\over{\partial t}}
       +{{i}\over{2}}(k''_1\pm b''_1)
            {{\partial^2 A^{b\mp}_{\omega_1}}\over{\partial t^2}}\cr
    &\hskip90pt
    =i{{\omega_1 p_1}\over{2cn_1}} A^{f\mp}_{\omega_3} A^{f\pm*}_{\omega_2}
        \exp(i(k_3-k_2+k_1)z\pm i(\beta_3+\beta_2+\beta_1)z),
  \eqdefn{eq:waveeq260}&\eqsubdef{eq:waveeq260a}\cr
    &{{\partial A^{f\pm}_{\omega_2}}\over{\partial z}}
       +(k'_2\mp a'_2) {{\partial A^{f\pm}_{\omega_2}}\over{\partial t}}
       +{{i}\over{2}}(k''_2\mp b''_2)
            {{\partial^2 A^{f\pm}_{\omega_2}}\over{\partial t^2}}\cr
    &\hskip90pt
    =i{{\omega_2 p_2}\over{2cn_2}} A^{f\mp}_{\omega_3} A^{b\mp*}_{\omega_1}
        \exp(i(k_3+k_1-k_2)z\pm i(\beta_3+\beta_2+\beta_1)z),
  &\eqsubdef{eq:waveeq260b}\cr
    &{{\partial A^{f\mp}_{\omega_3}}\over{\partial z}}
       +(k'_3\pm a'_3) {{\partial A^{f\mp}_{\omega_3}}\over{\partial t}}
       +{{i}\over{2}}(k''_3\pm b''_3)
            {{\partial^2 A^{f\mp}_{\omega_3}}\over{\partial t^2}}\cr
    &\hskip90pt
    =i{{\omega_3 p_3}\over{2cn_3}} A^{b\mp}_{\omega_1} A^{f\pm}_{\omega_2}
        \exp(-i(k_1-k_2+k_3)z\mp i(\beta_3+\beta_2+\beta_1)z).
  &\eqsubdef{eq:waveeq260c}\cr
  }
$$
We adopt the short-hand notations\numberedfootnote{Notice the difference to
  the usual co-propagating case of optical parametric amplification and
  oscillation, in which the electric dipolar part of the phase matching
  condition instead yields
  $$
    \Delta k=k_3-k_2-k_1\approx0.
  $$
  However, the nonlocal contribution $\Delta\alpha=\beta_3+\beta_2+\beta_1$
  is identical in both cases.}
$$
  \Delta k=k_3-k_2+k_1,\qquad\Delta\alpha=\beta_3+\beta_2+\beta_1,
  \eqdef{eq:waveeq270}
$$
and conclude the derivation of the equations for the field envelopes by
summarizing them as
$$
  \eqalignno{
    -{{\partial A^{b\mp}_{\omega_1}}\over{\partial z}}
       +(k'_1\pm a'_1) {{\partial A^{b\mp}_{\omega_1}}\over{\partial t}}
       +{{i}\over{2}}(k''_1\pm b''_1)
            {{\partial^2 A^{b\mp}_{\omega_1}}\over{\partial t^2}}
    &=i{{\omega_1 p_1}\over{2cn_1}} A^{f\mp}_{\omega_3} A^{f\pm*}_{\omega_2}
        \exp(i(\Delta k\pm \Delta\alpha)z),
  \eqdefn{eq:waveeq280}&\eqsubdef{eq:waveeq280a}\cr
    {{\partial A^{f\pm}_{\omega_2}}\over{\partial z}}
       +(k'_2\mp a'_2) {{\partial A^{f\pm}_{\omega_2}}\over{\partial t}}
       +{{i}\over{2}}(k''_2\mp b''_2)
            {{\partial^2 A^{f\pm}_{\omega_2}}\over{\partial t^2}}
    &=i{{\omega_2 p_2}\over{2cn_2}} A^{f\mp}_{\omega_3} A^{b\mp*}_{\omega_1}
        \exp(i(\Delta k\pm\Delta\alpha)z),
  &\eqsubdef{eq:waveeq280b}\cr
    {{\partial A^{f\mp}_{\omega_3}}\over{\partial z}}
       +(k'_3\pm a'_3) {{\partial A^{f\mp}_{\omega_3}}\over{\partial t}}
       +{{i}\over{2}}(k''_3\pm b''_3)
            {{\partial^2 A^{f\mp}_{\omega_3}}\over{\partial t^2}}
    &=i{{\omega_3 p_3}\over{2cn_3}} A^{b\mp}_{\omega_1} A^{f\pm}_{\omega_2}
        \exp(-i(\Delta k\pm\Delta\alpha)z).
  &\eqsubdef{eq:waveeq280c}\cr
  }
$$

\section{Continuous waves}
For continuous waves, the time derivatives of the envelopes vanish, and we are
left with the considerably simplified system
$$
  \eqalignno{
    -{{\partial A^{b\mp}_{\omega_1}}\over{\partial z}}
    &=i{{\omega_1 p_1}\over{2cn_1}} A^{f\mp}_{\omega_3} A^{f\pm*}_{\omega_2}
        \exp(i(\Delta k\pm \Delta\alpha)z),
  \eqdefn{eq:waveeq290}&\eqsubdef{eq:waveeq290a}\cr
    {{\partial A^{f\pm}_{\omega_2}}\over{\partial z}}
    &=i{{\omega_2 p_2}\over{2cn_2}} A^{f\mp}_{\omega_3} A^{b\mp*}_{\omega_1}
        \exp(i(\Delta k\pm\Delta\alpha)z),
  &\eqsubdef{eq:waveeq290b}\cr
    {{\partial A^{f\mp}_{\omega_3}}\over{\partial z}}
    &=i{{\omega_3 p_3}\over{2cn_3}} A^{b\mp}_{\omega_1} A^{f\pm}_{\omega_2}
        \exp(-i(\Delta k\pm\Delta\alpha)z).
  &\eqsubdef{eq:waveeq290c}\cr
  }
$$

\section{A modulated sign of the nonlinear coupling coefficient}
In quasi phase matching the idea is to periodically modulate the sign of the
nonlinear coupling coefficient, in order to reverse the phase of the optical
parametric process after having reached one coherence length of the interaction.
In this case, we assume a periodic switching of the sign of the coupling
coefficient without any other changes to its magnitude.

The Fourier decomposition of a function $s(z)$, which is defined by
$$
  s(z)=
  \cases{
    1, & if $0\le z<\Lambda/2$,\cr
    0, & if $\Lambda/2\le z<\Lambda$,\cr
  }
  \eqdef{eq:waveeq300}
$$
and is periodically extended by a period $\Lambda$, is given in a complex-valued
exponential representation by\numberedfootnote{Or equivalently in real-valued
  terms,
  $$
    s_M(z)={{1}\over{2}}+\sum^M_{m=1}{{(2/\pi)}\over{(2m-1)}}
      \sin\bigg({{2\pi(2m-1)}\over{\Lambda}}z\bigg),
  $$
  which, however, for the purpose of inclusion in the complex-valued wave
  equations is less suited.}
$$
  s_M(z)={{1}\over{2}}+\sum^M_{m=1}{{1}\over{i\pi(2m-1)}}
  \bigg[
    \exp\bigg(i{{2\pi(2m-1)}\over{\Lambda}}z\bigg)
      -\exp\bigg(-i{{2\pi(2m-1)}\over{\Lambda}}z\bigg)
  \bigg].
  \eqdef{eq:waveeq310}
$$
The periodic expansion and the various terms are shown in Fig.~2 below.
\bigskip
\centerline{\epsfxsize=340pt\epsfbox{python/graphs/boxcar.eps}}
\noindent
{\captionwide{\bf Figure~2.} Fourier expansion of the periodic expansion of
  the rectangular (boxcar) function, for terms up to and including $m=4$ and
  for a period of $\Lambda=2$ and a duty cycle of 0.5 (50\%).}
\bigskip
\noindent
Returning to the original purpose of this, we wish to construct a Fourier
decomposition of the nonlinear coupling coefficient of the form
$$
  p_j(z)=
  \cases{
    p_j, & if $0\le z<\Lambda/2$,\cr
    -p_j, & if $\Lambda/2\le z<\Lambda$,\cr
  }
  \eqdef{eq:waveeq320}
$$
that is to say, by using the definition above for the boxcar function $s(z)$,
$$
  p_j(z)=2(s(z)-1/2)p_j,
  \eqdef{eq:waveeq330}
$$
a $\Lambda$-periodic modulation of the sign of the nonlinear coupling
parameters $p_j(z)$ is expressed by the complex-valued Fourier decomposition
$$
  p_j(z)=2p_j\sum^M_{m=1}{{1}\over{i\pi(2m-1)}}
  \bigg[
    \exp\bigg(i{{2\pi(2m-1)}\over{\Lambda}}z\bigg)
      -\exp\bigg(-i{{2\pi(2m-1)}\over{\Lambda}}z\bigg)
  \bigg].
  \eqdef{eq:waveeq340}
$$
The terms of the Fourier decomposition of $p_j(z)$ are hence
$$
  \eqalignno{
    m&=1\qquad\rightarrow\qquad -i(2/\pi)
    \big[\exp(i{{2\pi z}/{\Lambda}})-\exp(-i{{2\pi z}/{\Lambda}})\big]p_j,
    \eqdefn{eq:waveeq350}&\eqsubdef{eq:waveeq350a}\cr
    m&=2\qquad\rightarrow\qquad -i({{2}/{3\pi}})
    \big[\exp(i{{6\pi z}/{\Lambda}})-\exp(-i{{6\pi z}/{\Lambda}})\big]p_j,
    &\eqsubdef{eq:waveeq350b}\cr
    m&=3\qquad\rightarrow\qquad -i({{2}/{5\pi}})
    \big[\exp(i{{10\pi z}/{\Lambda}})-\exp(-i{{10\pi z}/{\Lambda}})\big]p_j,
    &\eqsubdef{eq:waveeq350c}\cr
    m&=4\qquad\rightarrow\qquad -i({{2}/{7\pi}})
    \big[\exp(i{{14\pi z}/{\Lambda}})-\exp(-i{{14\pi z}/{\Lambda}})\big]p_j,
    &\eqsubdef{eq:waveeq350d}\cr
    m&=5\qquad\rightarrow\qquad \cdots\ ,
    &\eqsubdef{eq:waveeq350e}\cr
  }
$$
where the missing $m=0$ term stems from the fact that the sign-reversing form
of Eq.~\eqref{eq:waveeq330} is perfectly balanced around zero, in contrary to
the boxcar function as shown in Fig.~2.
Admittedly, things are now sorted out in a somewhat reversed order, but if
we allow for varying coupling coefficients $p_j(z)$ in~\eqref{eq:waveeq290},
only keeping the lowest-order term in Eq.~\eqref{eq:waveeq350a} of the
form~\eqref{eq:waveeq340} obtain
$$
  \eqalignno{
    -{{\partial A^{b\mp}_{\omega_1}}\over{\partial z}}
    &={{\omega_1 p_1}\over{\pi cn_1}} A^{f\mp}_{\omega_3} A^{f\pm*}_{\omega_2}
    \big[\exp(i{{2\pi z}/{\Lambda}})
      -\underbrace{\exp(-i{{2\pi z}/{\Lambda}})}_{\hbox{keep this}}\big]
        \exp(i(\Delta k\pm \Delta\alpha)z),
  \eqdefn{eq:waveeq360}&\eqsubdef{eq:waveeq360a}\cr
    {{\partial A^{f\pm}_{\omega_2}}\over{\partial z}}
    &={{\omega_2 p_2}\over{\pi cn_2}} A^{f\mp}_{\omega_3} A^{b\mp*}_{\omega_1}
    \big[\exp(i{{2\pi z}/{\Lambda}})
      -\underbrace{\exp(-i{{2\pi z}/{\Lambda}})}_{\hbox{keep this}}\big]
        \exp(i(\Delta k\pm\Delta\alpha)z),
  &\eqsubdef{eq:waveeq360b}\cr
    {{\partial A^{f\mp}_{\omega_3}}\over{\partial z}}
    &={{\omega_3 p_3}\over{\pi cn_3}} A^{b\mp}_{\omega_1} A^{f\pm}_{\omega_2}
    \big[
      \underbrace{\exp(i{{2\pi z}/{\Lambda}})}_{\hbox{keep this}}
        -\exp(-i{{2\pi z}/{\Lambda}})\big]
        \exp(-i(\Delta k\pm\Delta\alpha)z).
  &\eqsubdef{eq:waveeq360c}\cr
  }
$$
Keeping only the phase-matching terms with negative signs of the exponents of
$2\pi/\Lambda$, Eqs.~\eqref{eq:waveeq360} are reduced to
$$
  \eqalignno{
    -{{\partial A^{b\mp}_{\omega_1}}\over{\partial z}}
    &=-{{\omega_1 p_1}\over{\pi cn_1}} A^{f\mp}_{\omega_3} A^{f\pm*}_{\omega_2}
        \exp\big(i(\Delta k\pm \Delta\alpha - {{2\pi}/{\Lambda}})z\big),
    \eqdefn{eq:waveeq370}&\eqsubdef{eq:waveeq370a}\cr
    {{\partial A^{f\pm}_{\omega_2}}\over{\partial z}}
    &=-{{\omega_2 p_2}\over{\pi cn_2}} A^{f\mp}_{\omega_3} A^{b\mp*}_{\omega_1}
        \exp\big(i(\Delta k\pm\Delta\alpha - {{2\pi}/{\Lambda}})z\big),
    &\eqsubdef{eq:waveeq370b}\cr
    {{\partial A^{f\mp}_{\omega_3}}\over{\partial z}}
    &={{\omega_3 p_3}\over{\pi cn_3}} A^{b\mp}_{\omega_1} A^{f\pm}_{\omega_2}
        \exp\big(-i(\Delta k\pm\Delta\alpha - {{2\pi}/{\Lambda}})z\big).
    &\eqsubdef{eq:waveeq370c}\cr
  }
$$
We assign the short-hand notation
$$
  \kappa_{\pm}=\Delta k\pm\Delta\alpha - 2\pi/\Lambda
  \eqdef{eq:waveeq380}
$$
and clean up Eqs.~\eqref{eq:waveeq370} from signs, to produce the equation for
the spatial evolution of the idler ($\omega_1$), signal ($\omega_2$) and pump
($\omega_3$) envelopes as
$$
  \eqalignno{
    {{\partial A^{b\mp}_{\omega_1}}\over{\partial z}}
    &={{\omega_1 p_1}\over{\pi cn_1}} A^{f\mp}_{\omega_3} A^{f\pm*}_{\omega_2}
        \exp(i\kappa_{\pm} z),
    \eqdefn{eq:waveeq390}&\eqsubdef{eq:waveeq390a}\cr
    {{\partial A^{f\pm}_{\omega_2}}\over{\partial z}}
    &=-{{\omega_2 p_2}\over{\pi cn_2}} A^{f\mp}_{\omega_3} A^{b\mp*}_{\omega_1}
        \exp(i\kappa_{\pm} z),
    &\eqsubdef{eq:waveeq390b}\cr
    {{\partial A^{f\mp}_{\omega_3}}\over{\partial z}}
    &={{\omega_3 p_3}\over{\pi cn_3}} A^{b\mp}_{\omega_1} A^{f\pm}_{\omega_2}
        \exp(-i\kappa_{\pm} z).
    &\eqsubdef{eq:waveeq390c}\cr
  }
$$

\section{The concept of quasi phase matching}
In Eqs.~\eqref{eq:waveeq380} and~\eqref{eq:waveeq390}, we see how the concept
of the {\it coherence length} naturally appear in the theory of quasi phase
matching. If we for the moment drop the differential contribution
$\Delta\alpha$ to the phase, perfect phase matching of the OPA process
requires that
$$
  \kappa_{\pm}=\Delta k - 2\pi/\Lambda = 0.
  \eqsubdef{eq:qpm10}
$$
Since $\Lambda$ is the {\it period} of the modulation, containing two layers
of opposite sign for their respective nonlinear coupling coefficient, this
means that each layer of ``constant sign''\numberedfootnote{Keep in mind
  that we currently actually are looking at the first-order term of
  the Fourier decomposition in Eq.~\eqref{eq:waveeq350a}, which provides
  a {\it sinusoidal} variation of the coupling coefficient, rather than
  a periodically extended boxcar function. Hence the quotation marks.}
of the nonlinear coupling coefficient has the physical length $\Lambda/2$.
In other words, in order to provide perfect phase matching, or rather
{\it quasi phase matching}, each layer should have a physical thickness
of half this period, that is to say
$$
  L_{\rm C} = \Lambda/2 = \pi/\Delta k.
  \eqsubdef{eq:qpm20}
$$
This quantity is actually the very definition of the classically used
{\it coherence length} $L_{\rm C}$, which if we return to the form of
Eq.~\eqref{eq:waveeq380} including the differential contribution from
the nonlocal interaction yields\numberedfootnote{See the manuscript
  {\it Pulsed optical parametric amplification in chiral media} as
  submitted to J.~Opt.~Soc.~Am. B in March 2025.}
$$
  L^{\pm}_{\rm C} \equiv \pi/(\Delta k\pm\Delta\alpha).
  \eqsubdef{eq:qpm30}
$$
Thus, the concept of quasi phase matching can be summarized by that each layer
of constant sign of the nonlinear coupling coefficient (that is to say,
``$\chi^{(2)}$'') should be chosen of physical thickess corresponding to one
coherence length. In the case of an optically active medium, we have the
choice to design either for the LCP coherence length $L^+_{\rm C}$ or for
the RCP coherence length $L^-_{\rm C}$, providing different premises for
optimal phase matching.

\section{Solutions to the envelopes under the non-depleted pump approximation}
Whenever the pump may be considered as close to constant over the optical
parametric amplification, the system described by Eqs.~\eqref{eq:waveeq390}
is reduced to
%\numberedfootnote{Notice how this system very easily can be
%  cast into a normalized form by taking the spatial variable as
%  $\zeta=\gamma^{1/2}_2 A^{f\pm*}_{\omega_1}$ new variables
%  $u^{\pm*}_1=\gamma^{1/2}_2 A^{f\pm*}_{\omega_1}$ and
%  $u^{\pm*}_2=\gamma^{1/2}_2 A^{f\pm*}_{\omega_2}$, resulting in
%$$
%  \eqalign{
%    {{\partial A^{b\mp}_{\omega_1}}\over{\partial z}}
%      &=\gamma_1 A^{f\mp}_{\omega_3} A^{f\pm*}_{\omega_2}\exp(i\kappa_{\pm} z),\%cr
%    {{\partial A^{f\pm}_{\omega_2}}\over{\partial z}}
%      &=-\gamma_2 A^{f\mp}_{\omega_3} A^{b\mp*}_{\omega_1}\exp(i\kappa_{\pm} z),%\cr
%  }
%$$}
$$
  \eqalignno{
    {{\partial A^{b\mp}_{\omega_1}}\over{\partial z}}
      &=\gamma_1 A^{f\mp}_{\omega_3} A^{f\pm*}_{\omega_2}\exp(i\kappa_{\pm} z),
    \eqdefn{eq:waveeq390}&\eqsubdef{eq:waveeq390a}\cr
    {{\partial A^{f\pm}_{\omega_2}}\over{\partial z}}
      &=-\gamma_2 A^{f\mp}_{\omega_3} A^{b\mp*}_{\omega_1}\exp(i\kappa_{\pm} z),
    &\eqsubdef{eq:waveeq390b}\cr
  }
$$
where we defined the short-hand notation for the coupling coefficients
$$
  \gamma_j={{\omega_j p_j}\over{\pi cn_j}},\qquad j=1,2,
  \eqdef{eq:waveeq400}
$$
and where the pump envelope $A^{f\mp}_{\omega_3}=\hbox{const.}$, and where the idler
and signal are subject to the boundary conditions
$$
  A^{b\mp}_{\omega_1}(z=L)=0,\qquad
  A^{f\pm}_{\omega_2}(z=0)=A^{f\pm}_{\omega_2}(0).
  \eqsubdef{eq:waveeq410}
$$
By differentiating Eq.~\eqref{eq:waveeq390b} with respect to $z$ and
substituting Eq.~\eqref{eq:waveeq390a} for the idler into
Eq.~\eqref{eq:waveeq390b}, we get the ordinary differential equation
for the forward traveling signal ($\omega_2$) envelope as
$$
  \eqalign{
    {{\partial^2 A^{f\pm}_{\omega_2}}\over{\partial z^2}}
      &=-\gamma_2 A^{f\mp}_{\omega_3}
        {{\partial}\over{\partial z}}\big[A^{b\mp*}_{\omega_1}
          \exp(i\kappa_{\pm} z)\big]\cr
      &=-\gamma_2 A^{f\mp}_{\omega_3}
      \bigg(
        {{\partial A^{b\mp*}_{\omega_1}}\over{\partial z}}
          +i\kappa_{\pm} A^{b\mp*}_{\omega_1}
          \bigg)\exp(i\kappa_{\pm} z)\cr
      &=-\gamma_2 A^{f\mp}_{\omega_3}
      \bigg\{
        \gamma_1 A^{f\mp*}_{\omega_3} A^{f\pm}_{\omega_2}
        \exp(-i\kappa_{\pm} z)
          -i\kappa_{\pm}\bigg[
            {{\displaystyle{{\partial A^{f\pm}_{\omega_2}}
              \over{\partial z}}\exp(-i\kappa_{\pm} z)}
            \over{\gamma_2 A^{f\mp}_{\omega_3}}}\bigg]
          \bigg\}\exp(i\kappa_{\pm} z)\cr
      &=-\gamma_1\gamma_2 |A^{f\mp}_{\omega_3}|^2 A^{f\pm}_{\omega_2}
          +i\kappa_{\pm}{{\partial A^{f\pm}_{\omega_2}}\over{\partial z}}.\cr
  }
  \eqdef{eq:waveeq420}
$$
In similar, we by differentiating Eq.~\eqref{eq:waveeq390a} with respect
to $z$ and substituting Eq.~\eqref{eq:waveeq390b} for the idler into
Eq.~\eqref{eq:waveeq390a}, we get the ordinary differential equation for
the backward traveling idler ($\omega_1$) envelope as
$$
  \eqalign{
    {{\partial^2 A^{f\pm}_{\omega_1}}\over{\partial z^2}}
      &=\gamma_1 A^{f\mp}_{\omega_3}
        {{\partial}\over{\partial z}}\big[A^{f\pm*}_{\omega_2}
          \exp(i\kappa_{\pm} z)\big]\cr
      &=\gamma_1 A^{f\mp}_{\omega_3}
      \bigg(
        {{\partial A^{f\pm*}_{\omega_2}}\over{\partial z}}
          +i\kappa_{\pm} A^{f\pm*}_{\omega_2}
          \bigg)\exp(i\kappa_{\pm} z)\cr
      &=\gamma_1 A^{f\mp}_{\omega_3}
      \bigg\{
        -\gamma_2 A^{f\mp*}_{\omega_3} A^{b\mp}_{\omega_1}
        \exp(-i\kappa_{\pm} z)
          +i\kappa_{\pm}\bigg[
            {{\displaystyle{{\partial A^{b\mp}_{\omega_1}}
              \over{\partial z}}\exp(-i\kappa_{\pm} z)}
            \over{\gamma_1 A^{f\mp}_{\omega_3}}}\bigg]
          \bigg\}\exp(i\kappa_{\pm} z)\cr
      &=-\gamma_1\gamma_2 |A^{f\mp}_{\omega_3}|^2 A^{b\mp}_{\omega_1}
          +i\kappa_{\pm}{{\partial A^{b\mp}_{\omega_1}}\over{\partial z}}.\cr
  }
  \eqdef{eq:waveeq430}
$$
Thus, to summarize, we for the backward traveling idler and forward traveling
signal envelopes have the identical forms
\par\boxit{
$$
  \eqalignno{
    &{{\partial^2 A^{b\mp}_{\omega_1}}\over{\partial z^2}}
      -i\kappa_{\pm}{{\partial A^{b\mp}_{\omega_1}}\over{\partial z}}
      +\gamma_1\gamma_2 |A^{f\mp}_{\omega_3}|^2 A^{b\mp}_{\omega_1}=0,
    \eqdefn{eq:waveeq440}&\eqsubdef{eq:waveeq440a}\cr
    &{{\partial^2 A^{f\pm}_{\omega_2}}\over{\partial z^2}}
      -i\kappa_{\pm}{{\partial A^{f\pm}_{\omega_2}}\over{\partial z}}
      +\gamma_1\gamma_2 |A^{f\mp}_{\omega_3}|^2 A^{f\pm}_{\omega_2}=0.
    &\eqsubdef{eq:waveeq440b}\cr
  }
$$
}
\noindent
That these equations are identical, despite that one is for a backward traveling
wave and the other for a forward traveling component, should not be surprising
as they origin from the same wave equation, which always is invariant regardless
of the direction of propagation, with the solutions and directions determined by
the boundary conditions for launching of the waves.

\subsection{General solution}
In order to solve the second-order ordinary differential equation for the idler
and signal envelopes, we notice that they both share the characteristic
polynomial
$$
  r^2 - i\kappa_{\pm}r + \gamma_1\gamma_2 |A^{f\mp}_{\omega_3}|^2 =0,
  \eqdef{eq:waveeq450}
$$
with two distinct roots for each polarization state,
$$
  r_1 = i\kappa_{\pm}/2 + ib_{\pm},\qquad r_2 = i\kappa_{\pm}/2 - ib_{\pm},
  \eqdef{eq:waveeq460}
$$
where we defined
$$
  b_{\pm}=(\kappa^2_{\pm}/4 +\gamma_1\gamma_2|A^{f\mp}_{\omega_3}|^2)^{1/2}.
  \eqdef{eq:waveeq470}
$$
From the characteristic roots of Eq.~\eqref{eq:waveeq450}, we immediately find
the general solutions for envelopes of the idler signal as
$$
  \eqalignno{
    A^{b\mp}_{\omega_1}(z)
      &=C'_1\exp(i(\kappa_{\pm}/2+ib_{\pm})z)
          +C'_2\exp(i(\kappa_{\pm}/2-ib_{\pm})z)&\cr
      &=(C_1\cos(b_{\pm}z)+C_2\sin(b_{\pm}z))\exp(i\kappa_{\pm}z/2),
        \eqdefn{eq:waveeq480}&\eqsubdef{eq:waveeq480a}\cr
    A^{f\pm}_{\omega_2}(z)
      &=(D_1\cos(b_{\pm}z)+D_2\sin(b_{\pm}z))\exp(i\kappa_{\pm}z/2),
        &\eqsubdef{eq:waveeq480b}\cr
  }
$$
where $C_j$ and $D_j$, $j=1,2$, are constants of integration.

\subsection{Application of boundary conditions I -- The idler envelope}
Starting with the backward traveling idler, we from Eq.~\eqref{eq:waveeq480}
have that
$$
  (C_1\cos(b_{\pm}L)+C_2\sin(b_{\pm}L))\exp(i\kappa_{\pm}L/2)
    =A^{b\mp}_{\omega_1}(L)
    =\hbox{known}.
  \eqdefn{eq:waveeq490}
$$
Meanwhile, by differentiating Eq.~\eqref{eq:waveeq480} with respect to $z$,
evaluating the expression at $z=0$ and interpreting the result against
Eq.~\eqref{eq:waveeq390a}, we obtain a second relation for the coefficients
$C_1$ and $C_2$ as
$$
  \eqalign{
  &b_{\pm}(-C_1\sin(b_{\pm}0)+C_2\cos(b_{\pm}0))\exp(i\kappa_{\pm}0/2)
  \cr&\hskip60pt
  +i(\kappa_{\pm}/2)(C_1\cos(b_{\pm}0)+C_2\sin(b_{\pm}0))\exp(i\kappa_{\pm}0/2)
  \cr&\hskip100pt
    = \gamma_1 A^{f\mp}_{\omega_3} A^{f\pm*}_{\omega_2}(0)\exp(i\kappa_{\pm} 0),\cr
  }
  \eqdef{eq:waveeq500}
$$
that is to say,
$$
  i(\kappa_{\pm}/2)C_1+b_{\pm}C_2
    = \gamma_1 A^{f\mp}_{\omega_3}(0) A^{f\pm*}_{\omega_2}(0),
  \eqdef{eq:waveeq510}
$$
where we recall that this analysis entirely is done under the assumption
$A^{f\mp}_{\omega_3}(z)=\hbox{const.}$ Solving Eqs.~\eqref{eq:waveeq490}
and~\eqref{eq:waveeq500} for $C_1$ and $C_2$ then yields
\par\boxit{
$$
  \eqalignno{
    C_1&={{A^{b\mp}_{\omega_1}(L)\exp(-i\kappa_{\pm} L/2)
          -(\gamma_1/b_{\pm})\sin(b_{\pm}L)A^{f\mp}_{\omega_3}(0) A^{f\pm*}_{\omega_2}(0)}
            \over{\cos(b_{\pm}L)-i(\kappa_{\pm}/2b_{\pm})\sin(b_{\pm}L)}},
    \eqdefn{eq:waveeq520}&\eqsubdef{eq:waveeq520a}\cr
    C_2&={{(\gamma_1/b_{\pm})\cos(b_{\pm}L)A^{f\mp}_{\omega_3}(0) A^{f\pm*}_{\omega_2}(0)
          -i(\kappa_{\pm}/2b_{\pm})A^{b\mp}_{\omega_1}(L)\exp(-i\kappa_{\pm} L/2)}
            \over{\cos(b_{\pm}L)-i(\kappa_{\pm}/2b_{\pm})\sin(b_{\pm}L)}},
    &\eqsubdef{eq:waveeq520b}\cr
  }
$$
}\noindent
where we already at this stage took the opportunity to normalize the parameters
$b_{\pm}$, $\gamma_j$ and $\kappa_{\pm}$ by multiplying the numerators and
denominators in Eqs.~\eqref{eq:waveeq520} by the length $L$.
Thus, just to summarize without entering a rather messy algebra in attempting
to formulate an explicit form, the solution for the envelope of the backward
traveling idler wave from Eqs.~\eqref{eq:waveeq480a} and~\eqref{eq:waveeq520}
is
\par\boxit{
$$
  A^{b\mp}_{\omega_1}(z)=(C_1\cos(b_{\pm}z)+C_2\sin(b_{\pm}z))\exp(i\kappa_{\pm}z/2).
  \eqdef{eq:waveeq530}
$$
}

\subsection{Application of boundary conditions II -- The signal envelope}
Continuing with the forward traveling signal, we from Eq.~\eqref{eq:waveeq480}
at $z=0$ have that
$$
  (D_1\cos(b_{\pm}0)+D_2\sin(b_{\pm}0))\exp(i\kappa_{\pm}0/2)
    = D_1 = A^{f\pm}_{\omega_2}(0) =\hbox{known}.
  \eqdef{eq:waveeq540}
$$
In other words, the coefficient $D_1$ is trivially obtained as the amplitude
of the initial signal at $z=0$.
Meanwhile, in similar to the process of obtaining the coefficients for the
idler, by differentiating Eq.~\eqref{eq:waveeq480b} with respect to $z$,
evaluating the expression at $z=L$ and interpreting the result against
Eq.~\eqref{eq:waveeq390b} under the assumption that we do not launch an
idler at $z=L$ ($A^{b\mp*}_{\omega_1}(L)=0$), we obtain a second relation for
the coefficients $D_1$ and $D_2$ as
$$
  \eqalign{
  &b_{\pm}(-D_1\sin(b_{\pm}L)+D_2\cos(b_{\pm}L))\exp(i\kappa_{\pm}L/2)
  \cr&\hskip60pt
  +i(\kappa_{\pm}/2)(D_1\cos(b_{\pm}L)+D_2\sin(b_{\pm}L))\exp(i\kappa_{\pm}L/2)
  \cr&\hskip100pt
    = -\gamma_2 A^{f\mp}_{\omega_3} A^{b\mp*}_{\omega_1}(L)\exp(i\kappa_{\pm} L)
  \cr&\hskip100pt
    = 0,\cr
  }
  \eqdef{eq:waveeq550}
$$
that is to say, since we previously found that $D_1=A^{f\pm}_{\omega_2}(0)$, we
may solve Eq.~\eqref{eq:waveeq550} for $D_2$ to obtain\numberedfootnote{We may
  here ask ourselves the very basic question: {\it Why is it that the
  coupling coefficient $\gamma_2$ does not show up neither in
  Eq.~\eqref{eq:waveeq520}--\eqref{eq:waveeq530} nor Eq.~\eqref{eq:waveeq560}?}
  For example, $\gamma_1$ appear explicitly in Eq.~\eqref{eq:waveeq520},
  so why not $\gamma_1$? Is something wrong here? Obviously, the only way
  that $\gamma_2$ would enter the expressions explicitly is through
  Eq.~\eqref{eq:waveeq550}; however, there $\gamma_2$ is coupled to
  the fact that  $A^{b\mp*}_{\omega_1}(L)=0$, and $\gamma_2$ hence never
  enter the expressions at this stage. However, $\gamma_2$ is present
  through the definition of $b_{\pm}$ in Eq.~\eqref{eq:waveeq470},
  $b_{\pm}=(\kappa^2_{\pm}/4 +\gamma_1\gamma_2|A^{f\mp}_{\omega_3}|^2)^{1/2}$,
  so both coupling coefficients are actually present, hence solving
  this paradox.}
$$
  D_2={{\sin(b_{\pm}L)-i(\kappa_{\pm}/2b_{\pm})\cos(b_{\pm}L)}
      \over{\cos(b_{\pm}L)+i(\kappa_{\pm}/2b_{\pm})\sin(b_{\pm}L)}}
      A^{f\pm}_{\omega_2}(0),
  \eqdef{eq:waveeq560}
$$
To summarize, the envelope of the forward traveling signal is from
Eqs.~\eqref{eq:waveeq480b} and~\eqref{eq:waveeq540}--\eqref{eq:waveeq560}
explicitly given as
\par\boxit{
$$
    A^{f\pm}_{\omega_2}(z)
      =A^{f\pm}_{\omega_2}(0)\bigg(\cos(b_{\pm}z)+
      {{\sin(b_{\pm}L)-i(\kappa_{\pm}/2b_{\pm})\cos(b_{\pm}L)}
      \over{\cos(b_{\pm}L)+i(\kappa_{\pm}/2b_{\pm})\sin(b_{\pm}L)}}
      \sin(b_{\pm}z)
      \bigg)\exp(i\kappa_{\pm}z/2).
  \eqdef{eq:waveeq560}
$$
}
\noindent
As a simple sanity check on this expression, we see that it trivially fulfils
the initial condition when $z=0$.

\section{Signal amplification over a full passage of the medium}
When passing over the medium from $z=0$ to $z=L$, the LCP/RCP modes of the
signal expressed by Eq.~\eqref{eq:waveeq560} experience the gain
$$
  \eqalign{
  G^{\pm}_{\rm s}
    &\equiv{{|A^{f\pm}_{\omega_2}(L)|^2}\over{|A^{f\pm}_{\omega_2}(0)|^2}}\cr
    &=\bigg|\cos(b_{\pm}L)+
      {{\sin(b_{\pm}L)-i(\kappa_{\pm}/2b_{\pm})\cos(b_{\pm}L)}
      \over{\cos(b_{\pm}L)+i(\kappa_{\pm}/2b_{\pm})\sin(b_{\pm}L)}}
      \sin(b_{\pm}L)
      \bigg|^2\cr
    &=\bigg|{{\cos(b_{\pm}L)+i(\kappa_{\pm}/2b_{\pm})\sin(b_{\pm}L)}
      \over{\cos(b_{\pm}L)+i(\kappa_{\pm}/2b_{\pm})\sin(b_{\pm}L)}}
      \cos(b_{\pm}L)+
      {{\sin(b_{\pm}L)-i(\kappa_{\pm}/2b_{\pm})\cos(b_{\pm}L)}
      \over{\cos(b_{\pm}L)+i(\kappa_{\pm}/2b_{\pm})\sin(b_{\pm}L)}}
      \sin(b_{\pm}L)
      \bigg|^2\cr
    &=\bigg|
        {{{[\cos(b_{\pm}L)+i(\kappa_{\pm}/2b_{\pm})\sin(b_{\pm}L)]}\cos(b_{\pm}L)
          +{[\sin(b_{\pm}L)-i(\kappa_{\pm}/2b_{\pm})\cos(b_{\pm}L)]}\sin(b_{\pm}L)}
         \over{\cos(b_{\pm}L)+i(\kappa_{\pm}/2b_{\pm})\sin(b_{\pm}L)}}
      \bigg|^2\cr
    &={{1}\over{|\cos(b_{\pm}L)+i(\kappa_{\pm}/2b_{\pm})\sin(b_{\pm}L)|^2}}\cr
    &={{1}\over{\cos^2(b_{\pm}L)+(\kappa_{\pm}/2b_{\pm})^2\sin^2(b_{\pm}L)}}.\cr
  }
  \eqdef{eq:waveeq570}
$$
In perfect phase matching, $\kappa_{\pm}=0$ for whichever mode LCP/RCP, and the
gain under this circumstance takes the form
$$
  G^{\pm}_{\rm s}={{1}\over{\cos^2(b_{\pm}L)}},
  \eqdef{eq:waveeq580}
$$
which goes to infinity whenever
$$
  b_{\pm}L=\pi/2\quad\Rightarrow\quad G^{\pm}_{\rm s}\to\infty,
  \eqdef{eq:waveeq590}
$$
or from Eq.~\eqref{eq:waveeq470}, keeping in mind that $\kappa_{\pm}=0$,
equivalently
$$
  \gamma_1\gamma_2|A^{f\mp}_{\omega_3}|^2 L^2 = (\pi/2)^2
  \quad\Rightarrow\quad G^{\pm}_{\rm s}\to\infty.
  \eqdef{eq:waveeq600}
$$
In other words, the threshold is reached whenever the pump intensity reaches
$$
  \gamma_1\gamma_2|A^{f\mp}_{\omega_3}|^2=\Big({{\pi}\over{2L}}\Big)^2.
  \eqdef{eq:waveeq610}
$$
As the intensity is defined as $I=c_0n\varepsilon_0|E|^2$, where $c_0$ is the
vacuum speed of light, $n$ is the index of refraction of the medium and
$\varepsilon_0$ the vacuum permittivity, this means that the corresponding
LCP/RCP pump threshold intensities $I^{\pm}_{\rm th}$ expressed in SI units
become\numberedfootnote{We here ignore the fact that LCP/RCP in chiral
  media in fact have different refractive indices; also notice that we
  from Eq.~\eqref{eq:waveeq400} may reformulate this as the rather silly
  expression
  $$
    I^{\pm}_{\rm th}={{\pi^4 c^3 n_1 n_2 n_3\varepsilon_0}
                   \over{4\omega_1 \omega_2 p_1 p_2 L^2}}\quad\ldots
  $$}
$$
  I^{\pm}_{\rm th}={{cn_3\varepsilon_0}\over{\
      \gamma_1\gamma_2}}
  \Big({{\pi}\over{2L}}\Big)^2,
  \eqdef{eq:waveeq620}
$$

\section{Interpretation in normalized and dimensionless variables}
In the expressions for the explicit solutions and in the definition of
$b_{\pm}=(\kappa^2_{\pm}/4 +\gamma_1\gamma_2|A^{f\mp}_{\omega_3}|^2)^{1/2}$ in
Eq.~\eqref{eq:waveeq470}, we find that there are a set of naturally appearing
pairs of variables, such as
$$
  b_{\pm}L,\qquad
  \kappa_{\pm}L/2,\qquad
  \kappa_{\pm}/2b_{\pm},\qquad
  \hbox{and}\qquad\gamma_1\gamma_2|A^{f\mp}_{\omega_3}|^2L^2.
  \eqdef{eq:waveeq630}
$$
Let us therefore attempt to identify suitable normalized and dimensionless
forms involving these parameters as far as possible, starting with the pump
intensity $\gamma_1\gamma_2|A^{f\mp}_{\omega_3}|^2L^2$, which we via the threshold
of Eq.~\eqref{eq:waveeq620} find can be written as
$$
  \gamma_1\gamma_2|A^{f\mp}_{\omega_3}|^2L^2
    =\Big({{\pi}\over{2}}\Big)^2 {{I_{\rm pump}}\over{I_{\rm th}}}.
  \eqdef{eq:waveeq640}
$$
As for the term $\kappa_{\pm}L/2$, we should here be somewhat careful, as this
for the backward OPA contains $\Delta kL$ which for the backward configuration
is huge in numerical value as such; the idea in QPM is that this should be
very close to the term $2\pi L/\Lambda$, and we should hence separarate out
this pair as the effective phase mismatch.
Hence, the term $\kappa_{\pm}L/2$ can be formulated in terms of normalized and
dimensionless quotes as\numberedfootnote{See Eq.~\eqref{eq:waveeq380}
  for the definition of $\kappa_{\pm}$.}
$$
  \eqalign{
   \kappa_{\pm}L/2&\equiv(\Delta k\pm\Delta\alpha - 2\pi/\Lambda)L/2\cr
      &=\Big({{\Delta kL}\over{2}}-{{2\pi L}\over{2\Lambda}}\Big)
          \Big(
            1\pm{{\Delta\alpha}\over{\Delta k-2\pi/\Lambda}}
          \Big).\cr
  }
  \eqdef{eq:waveeq650}
$$
Meanwhile, the term $b_{\pm}L$ can similarly be formulated in terms of the
normalized and dimensionless quotes, including the pump intensity vs threshold,
as\numberedfootnote{Notice that at perfect quasi phase matching against the
  electric dipolar mismatch $\Delta k$, we have $2\pi/(\Delta k\Lambda)=1$,
  and hence the expression for $b_{\pm}$ becomes {\it symmetric} in the chiral
  phase mismatch parameter $\Delta\alpha$.}
$$
  \eqalign{
   b_{\pm}L&\equiv(\kappa^2_{\pm}/4 +\gamma_1\gamma_2|A^{f\mp}_{\omega_3}|^2)^{1/2}L\cr
      &=\big[(\kappa_{\pm}L/2)^2 +(\pi/2)^2(I_{\rm pump}/I_{\rm th})\big]^{1/2}\cr
      &=\big[((\Delta k\pm\Delta\alpha - 2\pi/\Lambda)L/2)^2
          +(\pi/2)^2(I_{\rm pump}/I_{\rm th})\big]^{1/2}\cr
      &=\bigg[\Big({{\Delta kL}\over{2}}-{{2\pi L}\over{2\Lambda}}\Big)^2
          \Big(
            1\pm{{\Delta\alpha}\over{\Delta k-2\pi/\Lambda}}
          \Big)^2
          +\Big({{\pi}\over{2}}\Big)^2{{I_{\rm pump}}\over{I_{\rm th}}}
        \bigg]^{1/2},\cr
  }
  \eqdef{eq:waveeq660}
$$
and finally and trivially,
$$
  \eqalign{
   {{\kappa_{\pm}}\over{2b_{\pm}}}=
   {{\kappa_{\pm}L/2}\over{b_{\pm}L}}
   &={{\displaystyle
        \Big({{\Delta kL}\over{2}}-{{2\pi L}\over{2\Lambda}}\Big)
          \Big(
            1\pm{{\Delta\alpha}\over{\Delta k-2\pi/\Lambda}}
        \Big)
   }\over{\displaystyle
      \bigg[\Big({{\Delta kL}\over{2}}-{{2\pi L}\over{2\Lambda}}\Big)^2
          \Big(
            1\pm{{\Delta\alpha}\over{\Delta k-2\pi/\Lambda}}
          \Big)^2
          +\Big({{\pi}\over{2}}\Big)^2{{I_{\rm pump}}\over{I_{\rm th}}}
      \bigg]^{1/2}}}.\cr
  }
  \eqdef{eq:waveeq670}
$$
Thus, by adopting the short-hand notation for the normalized and dimensionless
variables
\par\boxit{
$$
  \beta\equiv{{\Delta\alpha}\over{\Delta k-2\pi/\Lambda}},\qquad
  \delta\equiv\Big({{\Delta kL}\over{2}}-{{2\pi L}\over{2\Lambda}}\Big),\qquad
  \eta=({{\pi}/{2}})^2{{I_{\rm pump}}/{I_{\rm th}}},
  \eqdef{eq:waveeq672}
$$
}\noindent
the expressions $\kappa_{\pm}L/2$, $b_{\pm}L$ and ${{\kappa_{\pm}}/{2b_{\pm}}}$ are
compactified to
\par\boxit{
$$
  \kappa_{\pm}L/2=\delta(1\pm\beta),\qquad
  b_{\pm}L=(\delta^2(1\pm\beta)^2+\eta)^{1/2},\qquad
  {{\kappa_{\pm}}\over{2b_{\pm}}}={{\delta(1\pm\beta)}
    \over{(\delta^2(1\pm\beta)^2+\eta)^{1/2}}}.
  \eqdef{eq:waveeq674}
$$
}\noindent
In particular, in terms of these parameters the gain expressed by
Eq.~\eqref{eq:waveeq570} becomes
\par\boxit{
$$
  \eqalign{
  G^{\pm}_{\rm s}
    &={{1}\over{\cos^2(b_{\pm}L)+(\kappa_{\pm}/2b_{\pm})^2\sin^2(b_{\pm}L)}}.\cr
    &={{1}\over{\displaystyle\cos^2((\delta^2(1\pm\beta)^2+\eta)^{1/2})
    +\bigg({{\delta(1\pm\beta)}
            \over{(\delta^2(1\pm\beta)^2+\eta)^{1/2}}}
     \bigg)^2\sin^2((\delta^2(1\pm\beta)^2+\eta)^{1/2})}}.\cr
    &={{(\delta^2(1\pm\beta)^2+\eta)^{1/2}}
    \over{\displaystyle(\delta^2(1\pm\beta)^2+\eta)^{1/2}
    \cos^2((\delta^2(1\pm\beta)^2+\eta)^{1/2})
    +\delta^2(1\pm\beta)^2\sin^2((\delta^2(1\pm\beta)^2+\eta)^{1/2})}}.\cr
  }
  \eqdef{eq:waveeq676}
$$
}

\section{Interpretation in case of perfect QPM against electric dipolar
         phase mismatch}
Of particular interest is the interpretation of the terms $\kappa_{\pm}L/2$,
$b_{\pm}L$ and their quote $\kappa_{\pm}/2b_{\pm}$ in the case of perfect quasi
phase matching against the electric dipolar phase mismatch $\Delta k$, that is
to say whenever
$$
  \Delta k={{2\pi}\over{\Lambda}}.
  \eqdef{eq:perfectqpm10}
$$
In this particular case of interest,
$$
   \kappa_{\pm}L/2
      =\Big({{\Delta kL}\over{2}}\Big)
          \Big(
            1\pm{{\Delta\alpha}\over{\Delta k}}
               -\underbrace{{{2\pi}\over{\Delta k\Lambda}}}_{=1}
          \Big)
      =\pm{{\Delta\alpha L}\over{2}}
  \eqdef{eq:perfectqpm20}
$$
Meanwhile, the term $b_{\pm}L$ in the perfect QPM case becomes
$$
   b_{\pm}L
      =\bigg[\Big({{\Delta kL}\over{2}}\Big)^2
          \Big(
            1\pm{{\Delta\alpha}\over{\Delta k}}
              -\underbrace{{{2\pi}\over{\Delta k\Lambda}}}_{=1}
          \Big)^2
          +\Big({{\pi}\over{2}}\Big)^2{{I_{\rm pump}}\over{I_{\rm th}}}
        \bigg]^{1/2}
      =\bigg[
          \Big(
            {{\Delta\alpha L}\over{2}}
          \Big)^2
          +\Big({{\pi}\over{2}}\Big)^2{{I_{\rm pump}}\over{I_{\rm th}}}
        \bigg]^{1/2},
  \eqdef{eq:perfectqpm30}
$$
and finally and trivially,
$$
   {{\kappa_{\pm}}\over{2b_{\pm}}}
   =\pm{{\displaystyle\Big({{\Delta\alpha L}\over{2}}\Big)}
       \over{\displaystyle
        \bigg(
          \Big(
            {{\Delta\alpha L}\over{2}}
          \Big)^2
          +\Big({{\pi}\over{2}}\Big)^2{{I_{\rm pump}}\over{I_{\rm th}}}
        \bigg)^{1/2}
        }}
  \eqdef{eq:perfectqpm40}
$$
%We define the short-hand notation
%$$
%  \beta\equiv\Delta\alpha L/2,\qquad
%  \delta\equiv (\beta^2+\eta)^{1/2},\qquad
%  \eta=({{\pi}/{2}})^2{{I_{\rm pump}}/{I_{\rm th}}}.
%  \eqdef{eq:perfectqpm50}
%$$
%After some straightforward algebra, keeping in mind that with this notation
%$\kappa_{\pm}L/2=\pm\Delta\alpha L/2=\pm\beta$, $b_{\pm}L=\delta$, and
%$\kappa_{\pm}/2b_{\pm}=\pm\beta/(\beta^2+\eta)^{1/2}=\pm\beta/\delta$, the sign%al
%envelope from Eq.~\eqref{eq:waveeq560} may hence in the perfectly matched case
%against electric dipolar mismatch be written as
%\par\boxit{
%$$
%  \eqalign{
%    A^{f\pm}_{\omega_2}(z)
%%      &=A^{f\pm}_{\omega_2}(0)\bigg(\cos(b_{\pm}z)+
%%        {{\sin(b_{\pm}L)-i(\kappa_{\pm}/2b_{\pm})\cos(b_{\pm}L)}
%%        \over{\cos(b_{\pm}L)+i(\kappa_{\pm}/2b_{\pm})\sin(b_{\pm}L)}}
%%        \sin(b_{\pm}z)
%%        \bigg)\exp(i\kappa_{\pm}z/2)\cr
%      &=A^{f\pm}_{\omega_2}(0)\bigg(\cos(\delta z/L)+
%        {{\sin(\delta)\mp i(\beta/\delta)\cos(\delta)}
%        \over{\cos(\delta)\pm i(\beta/\delta)\sin(\delta)}}
%        \sin(\delta z/L)
%        \bigg)\exp(\pm i\beta z/L).\cr
%  }
%  \eqdef{eq:perfectqpm60}
%$$
%}
%\noindent
Now notice that in the fully quasi phase matched case against the electric
dipolar phase mismatch, $\kappa_{\pm}L/2$ is fully {\it antisymmetric} with
respect to chiral coefficient $\Delta\alpha$ for the phase mismatch, while
$b_{\pm}L$ is fully {\it symmetric} with respect to the same parameter.
Thus, in this particular case of perfect phase matching, the single-pass gain
is from Eq.~\eqref{eq:waveeq676} expressed as
$$
  \eqalign{
  G^{\pm}_{\rm s}
    &={{1}\over{\cos^2(b_{\pm}L)+(\kappa_{\pm}/2b_{\pm})^2\sin^2(b_{\pm}L)}}.\cr
    &=\Bigg\{
    \cos^2\Big(
          \Big(
            \Big(
              {{\Delta\alpha L}\over{2}}
            \Big)^2
            +\Big({{\pi}\over{2}}\Big)^2{{I_{\rm pump}}\over{I_{\rm th}}}
          \Big)^{1/2}
        \Big)
        \cr&\hskip60pt
    +{{\displaystyle\Big({{\Delta\alpha L}\over{2}}\Big)^2}
       \over{\displaystyle
        \bigg(
          \Big(
            {{\Delta\alpha L}\over{2}}
          \Big)^2
          +\Big({{\pi}\over{2}}\Big)^2{{I_{\rm pump}}\over{I_{\rm th}}}
        \bigg)
        }}
    \sin^2\Big(
          \Big(
            \Big(
              {{\Delta\alpha L}\over{2}}
            \Big)^2
            +\Big({{\pi}\over{2}}\Big)^2{{I_{\rm pump}}\over{I_{\rm th}}}
          \Big)^{1/2}
        \Big)
        \Bigg\}^{-1}.\cr
  }
  \eqdef{eq:perfectqpm70}
$$
This result might indeed be rather surprising, as this expression for the gain
is independent of the sign of $\Delta\alpha$; that is to say, the signal gain
$G_{\pm}$ is completely symmetric against the chiral phase mismatch
$\Delta\alpha$.
{\it In other words, the LCP and RCP modes are affected equally as functions
of the chiral phase mismatch parameter, without any distinction.}

\section{Reasonable parameter values}
In the backward OPA configuration, the phase mismatch parameters are radically
different than in the classical forward configuration. To start with, the
expressions in Eqs.~\eqref{eq:waveeq650}--\eqref{eq:waveeq670} all contain
the electric dipolar phase mismatch factor $\Delta kL/2$ which from
Eq.~\eqref{eq:waveeq270} yields
$$
  {{\Delta kL}\over{2}}\equiv(k_3-k_2+k_1)L/2,
  \qquad\Delta\alpha=\beta_3+\beta_2+\beta_1,
\eqdef{eq:waveeq680}
$$
where $k_j=n_j\omega_j/c=n_j2\pi/\lambda_0$ are the magnitude of the
corresponding wave vectors.
If the idler and signal are of comparable wavelengths, say in the order of
1200 nm, and the pump (fulfilling the requirement $\omega_3=\omega_2+\omega_1$)
in the order of 600 nm, and that we furthermore assume the refractive index
for the idler and signal being in the order of $n_{1,2}\sim1.7$ and the pump
$n_{1,2}\sim1.9$, we then from Eq.~\eqref{eq:waveeq680} obtain the typical
order of the phase mismatch over a crystal of length $L=10$ mm as
$$
  \eqalign{
  {{\Delta kL}\over{2}}
    &={{1}\over{2}}\Big(
        {{2\pi\times 1.9}\over{600\times10^{-9}\ {\rm m}}}
          -{{2\pi\times 1.7}\over{1200\times10^{-9}\ {\rm m}}}
          +{{2\pi\times 1.7}\over{1200\times10^{-9}\ {\rm m}}}
      \Big)\times(10\times10^{-3}\ {\rm m})
    \approx 99\times10^3.\cr
  }
  \eqdef{eq:waveeq690}
$$
In order to match this with quasi phase matching configuration, we from
Eq.~\eqref{eq:waveeq650} hence require that the spatial period $\Lambda$
is chosen as
$$
  \eqalign{
    1\pm{{\Delta\alpha}\over{\Delta k}}-{{2\pi}\over{\Delta k\Lambda}}=0
    \quad\Leftrightarrow\quad
    \Lambda={{2\pi}
             \over{\displaystyle\Delta k\Big(1\pm{{\Delta\alpha}
                                                  \over{\Delta k}}\Big)}}
    \quad\Leftrightarrow\quad
    \Lambda={{\pi L}
             \over{\displaystyle\Big({{\Delta kL}\over{2}}\Big)
               \Big(1\pm{{\Delta\alpha}\over{\Delta k}}\Big)}},
  }
  \eqdef{eq:waveeq700}
$$
that is to say, if we design this period to just match the electric dipolar
mismatch $\Delta k$, ignoring $\Delta\alpha/\Delta k$,
$$
  \eqalign{
    \Lambda={{\pi (10\times10^{-3}\ {\rm m})}\over{(99\times10^3)}}
    \approx 0.32\times10^{-6}\ {\rm m} = 320\ {\rm nm}.
  }
  \eqdef{eq:waveeq710}
$$
Thus, we may at first conclude that a reasonable value for the dimensionless
``direct'' electric dipolar phase mismatch parameter is
$$
  {{\Delta kL}\over{2}}=100\times10^3.
  \eqdef{eq:waveeq720}
$$
Thus, in order to achieve phase matching as described by $\kappa_{\pm}=0$ in
Eq.~\eqref{eq:waveeq650},\numberedfootnote{Eq.~\eqref{eq:waveeq650} simply
states that
  $$
    \kappa_{\pm}L/2=\Big({{\Delta kL}\over{2}}\Big)
       \Big(
         1\pm{{\Delta\alpha}\over{\Delta k}}-{{2\pi}\over{\Delta k\Lambda}}
       \Big),
  $$
  which should equal to zero at perfect phase matching.}
we have three primary choices for the quasi phase matching  period $\Lambda$:
$$
  \eqalign{
    {{2\pi}\over{\Delta k\Lambda}}&=1+{{\Delta\alpha}\over{\Delta k}}
    \qquad\Rightarrow\qquad\kappa_+=0 \hskip45pt\hbox{(LCP phase matched)}\cr
    {{2\pi}\over{\Delta k\Lambda}}&=1-{{\Delta\alpha}\over{\Delta k}}
    \qquad\Rightarrow\qquad\kappa_-=0 \hskip45pt\hbox{(RCP phase matched)}\cr
    {{2\pi}\over{\Delta k\Lambda}}&=1
    \hskip50pt\Rightarrow\qquad\kappa_{+}+\kappa_{-}=0
    \qquad\hbox{(phase matched at zero gyrotropy)}\cr
  }
$$

\section{Graphs and visual interpretations}
We will in the following focus on the dependence of backward-wave optical
parametric amplification on the electric dipolar phase mismatch against the
nominal quasi phase matching period $\Lambda$, using the normalized set of
parameters as defined in Eq.~\eqref{eq:waveeq672},
$$
  \beta\equiv{{\Delta\alpha}\over{\Delta k-2\pi/\Lambda}},\qquad
  \delta\equiv\Big({{\Delta kL}\over{2}}-{{2\pi L}\over{2\Lambda}}\Big),\qquad
  \eta=({{\pi}/{2}})^2{{I_{\rm pump}}/{I_{\rm th}}}.
  \eqdef{eq:graphs10}
$$

\bigskip
\centerline{\epsfxsize=240pt\epsfbox{python/graphs/graph-01-delta-0.50.eps}}
\medskip
\noindent
{\bf Figure~3}. Stokes parameters for the forward traveling signal envelope,
for the case of an electric dipolar phase mismatch of
$(\Delta k=2\pi/\Lambda)L/2=0.5$ and with a pump intensity to threshold
quote of $\eta=(\pi/2)^2I_{\rm pump}/I_{\rm th}=2.0$.
[Filename: {\tt python/graphs/graph-01-delta-0.50.eps}]
\vfill\eject
\bigskip
\centerline{\epsfxsize=240pt\epsfbox{python/graphs/graph-01-delta-1.00.eps}}
\medskip
\noindent
{\bf Figure~4}. Identical to Fig.~3 but with an electric dipolar phase mismatch
against nominal QPM period $(\Delta k=2\pi/\Lambda)L/2=1.0$.
[Filename: {\tt python/graphs/graph-01-delta-1.00.eps}]
\bigskip
\centerline{\epsfxsize=240pt\epsfbox{python/graphs/graph-01-delta-1.50.eps}}
\medskip
\noindent
{\bf Figure~5}. Identical to Fig.~3 but with an electric dipolar phase mismatch
against nominal QPM period $(\Delta k=2\pi/\Lambda)L/2=1.5$.
[Filename: {\tt python/graphs/graph-01-delta-1.50.eps}]
\vfill\eject

\centerline{\epsfxsize=310pt\epsfbox{poincare/graph-01-poincare.eps}}
\medskip
\noindent
{\bf Figure~6}. Poincar\'e map of the Stokes trajectories described by
Figs.~1--3.
[Filename: {\tt poincare/graph-01-poincare.eps}]
\bigskip
\bigskip
\centerline{\epsfxsize=250pt\epsfbox{python/graphs/graph-02.eps}}
\medskip
\noindent
{\bf Figure~7}. Signal gain $G_{\pm}$ as function of the chiral contribution
  $\Delta\alpha$ to the phase mismatch, for the case when the quasi phase
  matching period $\Lambda$ is chosen to perfectly match the electric dipolar
  phase mismatch, that is to say with $\Delta k=2\pi/\Lambda$. The gain
  curves are here mapped for a set of pump intensity quotes vs threshold,
  $\eta=(\pi/2)^I_{\rm pump}/I_{\rm th}$. Notice the completely symmetric dependence
  of gain against $\Delta\alpha$.
[Filename: {\tt python/graphs/graph-02.eps}]
\vfill\eject

\centerline{\epsfxsize=240pt\epsfbox{python/graphs/graph-03-surf.eps}}
\medskip
\noindent
{\bf Figure~8}. The signal gain $G_{+}$ (LCP, orange surface) and $G_{-}$
(RCP, blue surface) mapped as functions of the electric dipolar phase
mismatch $\delta=(\Delta k-2\pi/\Lambda)L/2$, against the nominal quasi
phase matching period $\Lambda$, and the chiral contribution to the phase
matching $\beta=\Delta\alpha/(\Delta k-2\pi/\Lambda)$.
Levels of constant gain are at the bottom plane mapped for $G_{+}$ (LCP,
orange contours) and $G_{-}$ (RCP, blue contours).
[Filename: {\tt python/graphs/graph-03-surf.eps}]
\bigskip
\centerline{\epsfxsize=240pt\epsfbox{python/graphs/graph-03-gplus-image.eps}}
\medskip
\noindent
{\bf Figure~9}.
Constant levels of the signal gain $G_+$ (LCP), showing the dependence of the
electric dipolar phase mismatch $\delta=(\Delta k-2\pi/\Lambda)L/2$ relative
the chiral contribution to the phase matching
$\beta=\Delta\alpha/(\Delta k-2\pi/\Lambda)$.
%\vfill\eject
[Filename: {\tt python/graphs/graph-03-gplus-image.eps}]
\vfill\eject

\centerline{\epsfxsize=240pt\epsfbox{python/graphs/graph-03-gminus-image.eps}}
\medskip
\noindent
{\bf Figure~10}.
Same as Fig.~9 but for $G_-$ (RCP), being antisymmetric to the LCP case with
respect to the chirality mismatch term $\Delta\alpha$.
[Filename: {\tt python/graphs/graph-03-gminus-image.eps}]
\bigskip

\centerline{\epsfxsize=240pt\epsfbox{python/graphs/graph-04-s0-surface.eps}}
\medskip
\noindent
{\bf Figure~11}.
Map of the intensity gain $S_0(L)/S_0(0)$ vs electric dipolar phase mismatch
$\delta=(\Delta k-2\pi/\Lambda)L/2$ and the chiral contribution to the phase
matching $\beta=\Delta\alpha/(\Delta k-2\pi/\Lambda)$. The map is constructed
for a pump intensity to threshold quote of
$\eta=(\pi/2)^2I_{\rm pump}/I_{\rm th}=2.0$. Contours at constant intensity gain
are mapped at the bottom plane.
[Filename: {\tt python/graphs/graph-04-s0-surface.eps}]
\vfill\eject

\centerline{\epsfxsize=240pt\epsfbox{python/graphs/graph-04-s0-5.00-image.eps}}
\medskip
\noindent
{\bf Figure~12}.
Map of the levels of constant intensity gain $S_0(L)/S_0(0)$, showing the
complex interconnection between the electric dipolar phase mismatch
$\delta=(\Delta k-2\pi/\Lambda)L/2$ and the chiral contribution to the phase
matching $\beta=\Delta\alpha/(\Delta k-2\pi/\Lambda)$. The mapped contours
at constant intensity gain are equal to those mapped in Fig.~8, for a pump
intensity to threshold quote of $\eta=(\pi/2)^2I_{\rm pump}/I_{\rm th}=2.0$.
[Filename: {\tt python/graphs/graph-04-s0-5.00-image.eps}]
\bigskip

\centerline{\epsfxsize=240pt\epsfbox{python/graphs/graph-04-s0-contour-5.00-black.eps}}
\medskip
\noindent
{\bf Figure~13}.
Identical to Fig.~12 but in plain black and white, possibly more suitable for
printing in a journal.
[Filename: {\tt python/graphs/graph-04-s0-contour-5.00-black.eps}]
\vfill\eject

\section{Signal ellipticity dependence on electric dipolar and chiral
         phase mismatch, pump intensity-to-threshold quote 2.0}
\bigskip
\bigskip
\centerline{\epsfxsize=240pt\epsfbox{python/graphs/graph-05-s3-2.00-surface.eps}}
\medskip
\noindent
{\bf Figure~14}.
Map of the transmitted signal ellipticity $S_3(L)/S_0(L)$ vs electric dipolar phase
mismatch $\delta=(\Delta k-2\pi/\Lambda)L/2$ and the chiral contribution to
the phase matching $\beta=\Delta\alpha/(\Delta k-2\pi/\Lambda)$.
The polarization state of the signal and pump at $z=0$ were linearly polarized,
and the map is constructed for a pump intensity to threshold quote of
$\eta=(\pi/2)^2I_{\rm pump}/I_{\rm th}=2.0$. Contours at constant intensity gain
are mapped at the bottom plane.
[Filename: {\tt python/graphs/graph-05-s3-2.00-surface.eps}]
\bigskip

\centerline{\epsfxsize=240pt\epsfbox{python/graphs/graph-05-s3-2.00-image.eps}}
\medskip
\noindent
{\bf Figure~15}.
Map of levels of constant transmitted signal ellipticity $S_3(L)/S_0(L)$,
showing the interconnection between electric dipolar phase mismatch
$\delta=(\Delta k-2\pi/\Lambda)L/2$ and the chiral contribution to the
phase matching $\beta=\Delta\alpha/(\Delta k-2\pi/\Lambda)$.
The polarization state of the signal and pump at $z=0$ were linearly polarized,
and the map is constructed for a pump intensity to threshold quote of
$\eta=(\pi/2)^2I_{\rm pump}/I_{\rm th}=2.0$.
These contours are identical to the ones mapped at the bottom plane of Fig.~14.
[Filename: {\tt python/graphs/graph-05-s3-2.00-image.eps}]
\vfill\eject

\centerline{\epsfxsize=240pt\epsfbox{python/graphs/graph-05-s3-2.00-contour-black.eps}}
\medskip
\noindent
{\bf Figure~16}.
Levels identical to the ones shown in Fig.~15, but in plain black and white,
suitable for printing.
[Filename: {\tt python/graphs/graph-05-s3-2.00-contour-black.eps}]
\vfill\eject

\section{Signal ellipticity dependence on electric dipolar and chiral
         phase mismatch, pump intensity-to-threshold quote 5.0}
\bigskip
\bigskip
\centerline{\epsfxsize=240pt\epsfbox{python/graphs/graph-05-s3-5.00-surface.eps}}
\medskip
\noindent
{\bf Figure~17}.
Map of the transmitted signal ellipticity $S_3(L)/S_0(L)$ vs electric dipolar phase
mismatch $\delta=(\Delta k-2\pi/\Lambda)L/2$ and the chiral contribution to
the phase matching $\beta=\Delta\alpha/(\Delta k-2\pi/\Lambda)$.
The polarization state of the signal and pump at $z=0$ were linearly polarized,
and the map is constructed for a pump intensity to threshold quote of
$\eta=(\pi/2)^2I_{\rm pump}/I_{\rm th}=3.0$. Contours at constant intensity gain
are mapped at the bottom plane.
[Filename: {\tt python/graphs/graph-05-s3-5.00-surface.eps}]
\bigskip

\centerline{\epsfxsize=240pt\epsfbox{python/graphs/graph-05-s3-5.00-image.eps}}
\medskip
\noindent
{\bf Figure~18}.
Map of levels of constant transmitted signal ellipticity $S_3(L)/S_0(L)$,
showing the interconnection between electric dipolar phase mismatch
$\delta=(\Delta k-2\pi/\Lambda)L/2$ and the chiral contribution to the
phase matching $\beta=\Delta\alpha/(\Delta k-2\pi/\Lambda)$.
The polarization state of the signal and pump at $z=0$ were linearly polarized,
and the map is constructed for a pump intensity to threshold quote of
$\eta=(\pi/2)^2I_{\rm pump}/I_{\rm th}=2.0$.
These contours are identical to the ones mapped at the bottom plane of Fig.~17.
[Filename: {\tt python/graphs/graph-05-s3-5.00-image.eps}]
\vfill\eject

\centerline{\epsfxsize=240pt\epsfbox{python/graphs/graph-05-s3-5.00-contour-black.eps}}
\medskip
\noindent
{\bf Figure~19}.
Levels identical to the ones shown in Fig.~18, but in plain black and white,
suitable for printing.
[Filename: {\tt python/graphs/graph-05-s3-5.00-contour-black.eps}]
\vfill\eject

\bye
